
%% -*- latex -*-
\newtoks\nicsroot\directlua{ tex.settoks("nicsroot", os.getenv("NICS_ROOT") or error("NICS_ROOT environment variable has to be set, use the Makefile")) }
\input{\the\nicsroot /src/nics-cached.tex}
\endofdump
\input{\the\nicsroot /src/nics-noncached.tex}
\hypersetup{
  pdfauthor={Mihaly Barasz (Nilcons)},
  pdftitle={Secure C++ Programming},
}
\nicsgrid=0

\begin{document}

%% \nicstitleslide{images/chain-laptop}{Secure C++ Programming}{Ide szeretnék valami jobbat (könyv és telefon nélkül) vagy csak szebbet}
%% \nicstitleslide{images/asm-puzzle}{Assembly Basics}{Hello World in x86/x86-64 Asm}
%% \nicstitleslide{images/workshop}{Assembly Hello World}{Workshop | Ide lehetne olyan, ahol puzzle-t összeraknak}
%% \nicstitleslide{images/overflow-waterfall}{Stack Buffer Overflows}{Introduction | Valami jobban ``túlcsordulós'' | flood, spill, dam}
%% \nicstitleslide{images/stack-smashing}{Stack Smashing}{Workshop}
%% \nicstitleslide{images/hardhats}{\mono{W^X}}{Executable-space protection}
%% \nicstitleslide{images/stack-coins}{Stack Protector}{Detect stack smashing}
%% \nicstitleslide{images/dice}{ASLR}{Address Randomization}
%% \nicstitleslide{images/ro-maci}{RELRO}{Relocation Read Only | Valami, ami jobban kifér?}
%% \nicstitleslide{images/heap-overflow}{Heap Overflows}{}
%% \nicstitleslide{images/format-string}{Format String Vulnerabilities}{}
%% \nicstitleslide{images/question}{Integer Vulnerabilities}{Valami (egész) számok?}
%% \nicstitleslide{images/race}{Concurrency}{}
%% \nicstitleslide{images/best-practices}{Best Practices}{}
%% \nicstitleslide{images/question}{Modern C++}{Valami modern épület?}

%% \nicstitleslide{images/thankyou}{The End}{Thanks, and good luck with all your coding!}





\nicstitleslide[Scanrail1/shutterstock]{images/titlepage}{Secure C++ Programming}{Dublin, May 13--14, 2019}

\section{Welcome}

\begin{slide}{Introductions}{}
  \begin{nicscolumn}
    \nicsheader{Each person please say:}
    \nicsitem{Their name}
    \nicsitem{What is their background}
    \nicsitem{How much C/C++ experience they have}
    \nicsitem{The main thing they want to learn}
    \nicsmedskip
    \nicspar{Me: Mihály Bárász <mihaly.barasz@nilcons.com>}
  \end{nicscolumn}
\end{slide}

\begin{slide}{Schedule}{}
  \begin{nicscolumn*}{1cm}{1.5cm}{8cm}
    \nicsheader{Today}
    \nicsitem{Assembly}
    \nicsitem{Basic vulnerabilities}
    \nicsitem{Some philosophy}
    \nicsitem{Most important generic protections}
    \nicsitem{Down-to-the-metal exercises}
    \nicsmedskip
    \nicsheader{Tomorrow}
    \nicsitem{More vulnerabilities}
    \nicsitem{Best practices}
    \nicsitem{More philosophy}
    \nicsitem{More exercises}
  \end{nicscolumn*}
  \begin{nicscolumn*}{9.5cm}{2cm}{5.5cm}
    \begin{nicsextern}{}
      \nicsincludepic{images/schedule}
    \end{nicsextern}
  \end{nicscolumn*}
\end{slide}



\section{Assembly}
\nicstitleslide{images/asm-puzzle}{Assembly Basics}{Hello World in x86/x86-64 Asm}

\begin{slide}{Resources}{Learning & Reference}
  \begin{nicscolumn}
    \nicsheader{Online references}
    \nicsitem{\url{https://www.felixcloutier.com/x86/}}
    %\nicsitem{\url{https://www.aldeid.com/wiki/X86-assembly/Instructions}\\ (less comprehensive, more readable)}
    \nicsitem{\url{https://godbolt.org/}}
    \nicsitem{\url{https://defuse.ca/online-x86-assembler.htm}\\(online assembler/disassembler)}
    \nicsitem{\url{http://www.sandpile.org/}}
    \nicsheader{Learning, Books}
    \nicsitem{\url{https://en.wikibooks.org/wiki/X86_Assembly}}
    \nicsitem{Programming from the Ground Up, Jonathan Bartlett\\
      old, 32 bit, but very good \url{https://savannah.nongnu.org/projects/pgubook/}}
  \end{nicscolumn}
\end{slide}

\begin{slide}{Syntax}{AT&T vs Intel syntax}
  \begin{nicscolumn*}{0.7cm}{2cm}{7.2cm}
    \nicsitem{Common on Linux, default syntax for most Linux tools:
      \mono{gcc}, \mono{as}, \mono{gdb}\dots\\ ~ \\}
    \nicsitem{\mono{mov \%rax, \%rbx}\\
      Order: source, destination\\
      Registers prefixed with \%}
    \nicsitem{\mono{mov $0x41, \%rax}\\
      Constants (immediates) prefixed with \$;
      C syntax for hexadecimals
    }
  \end{nicscolumn*}
  \begin{nicscolumn*}{8.3cm}{2cm}{7cm}
    \nicsitem{Universal on Windows, used in many popular Linux tools
      (\mono{\{n,y,f\}asm}), can be switched to in most tools \\ ~}
    \nicsitem{\mono{mov rbx, rax}\\Order: destination, source\\}
    \nicsitem{\mono{mov rax, 41h}\\
      Special syntax for hexadecimals (often \mono{0x} can be used too)
    }
  \end{nicscolumn*}
\end{slide}

\begin{slide}{Syntax\dots}{AT&T vs Intel}
  \begin{nicscolumn*}{0.7cm}{1.5cm}{7cm}
    \nicsitem{\mono{mov $symbol, \%rax}\\
      Loads the address of \mono{symbol}}
    \nicsitem{\mono{mov symbol, \%rax}\\
      Loads the content of \mono{symbol}\\}
    \nicspar{Memory references:}
    \nicsitem{\mono{mov 0x18(\%rdi,\%rsi,4), \%rax}\\
      {\tiny displacement(base register, offset register, scalar)}}
    \nicsitem{\mono{movq $5, (\%rax)}\\
      \mono{movb $5, (\%rax)}\\
      \mono{b}, \mono{w}, \mono{l}, \mono{q} suffix to the opcode}
  \end{nicscolumn*}
  \begin{nicscolumn*}{8.3cm}{1.5cm}{7.5cm}
    \nicsitem{\mono{mov rax, symbol}\\ ~}
    \nicsitem{\mono{mov rax, [symbol]}\\ ~ \\}
    \nicspar{}
    \nicsitem{\mono{mov rax, [rdi + 18h + 4*rsi]}\\
      {\tiny [base register + displacement + scalar*offset register]}}
    \nicsitem{\mono{mov qword [rax], 5}\\
      \mono{mov byte [rax], 5} \\
      \mono{byte}, \mono{word}, \mono{dword}, \mono{qword} prefix \\ for the operand}
  \end{nicscolumn*}
\end{slide}

\begin{slide}{Syntax\dots}{AT&T vs Intel}
  \begin{nicscolumn*}{0.7cm}{1.5cm}{7cm}
    \nicspar{Mnemonic differences}
    \nicsitem{\mono{cwtd}, \mono{cdtl}, \mono{cqto}, \dots\\}
    \nicsitem{Comments prefixed with \mono{\#}}
    \nicsitem{Directives prefixed with \mono{.}}
  \end{nicscolumn*}
  \begin{nicscolumn*}{8.3cm}{1.5cm}{7.5cm}
    \nicspar{}
    \nicsitem{\mono{cwd}, \mono{cdq}, \mono{cqo}, \dots\\}
    \nicsitem{Comments prefixed with \mono{;}}
    \nicsitem{Directives have no prefix}
  \end{nicscolumn*}
\end{slide}

\begin{slide}{Syntax example}{AT&T vs Intel}
  \begin{nicscolumn*}{4cm}{1.5cm}{8cm}
    \nicsexterncode{c}
    \begin{nicsextern}[width=5cm]{}
      int example(long long a, int *b) {
        return a / b[5];
      }
    \end{nicsextern}
  \end{nicscolumn*}
  \begin{nicscolumn*}{0.7cm}{3cm}{7cm}
    \nicsexterncode{asm}
    \begin{nicsextern}[height=3.7cm]{}
      .section .text

      .global example
      example:
        # First argument to %rax and sign extend to %rdx
        movq    %rdi, %rax
        cqto
        # b[5] sign extended to %rcx
        movslq  20(%rsi), %rcx
        idivq   %rcx
        # Result is in %rax where we need it
        ret
    \end{nicsextern}
  \end{nicscolumn*}
  \begin{nicscolumn*}{8.3cm}{3cm}{7.5cm}
    \nicsexterncode{asm}
    \begin{nicsextern}[height=3.7cm]{}
      section .text

      global example
      example:
        ; First argument to rax and sign extend to rdx
        mov    rax, rdi
        cqo
        ; b[5] sign extended to rcx
        movsxd rcx, dword [rsi+20]
        idiv   rcx
        ; Result is in rax where we need it
        ret
    \end{nicsextern}
  \end{nicscolumn*}
\end{slide}

\nicstitleslide{images/workshop}{Assembly Hello World}{Workshop}

\begin{slide}{Assembly Workshop}{Hello World!}
  \begin{nicscolumn}
    \nicsheader{Goal: get comfortable with the environment and assembly on Linux}
    \nicspar{Let's write following simple programs:}
    \nicsitem{Exit with some fix exit code (no crashing!)}
    \nicsitem{Exit with the exit code equal to number of command line arguments \\
      Hint: debug and inspect the top of the stack}
    \nicsitem{Classic ``Hello World''}
    \nicspar{Additionally:}
    \nicsitem{Try both syntaxes}
    \nicsitem{Try 32 and 64 bit executables}
    \nicsitem{Learn how to get just the actual code part}
  \end{nicscolumn}
\end{slide}

\begin{slide}{Assembly Workshop\dots}{}
  \begin{nicscolumn*}{0.7cm}{1.5cm}{7cm}
    % TODO(klao): a slide with the regular ABI
    % References
    \nicsheader{Linux syscall ABI}
    \nicspar{64 bit}
    \nicsitem{Syscall number in \mono{rax}}
    \nicsitem{Arguments: \mono{rdi}, \mono{rsi}, \mono{rdx}, \mono{r10}, \mono{r8}, \mono{r9}}
    \nicsitem{Instruction: \mono{syscall}}
    \nicsitem{Result in \mono{rax}}
    \nicsitem{Note: clobbers \mono{rcx} and \mono{r11}}
  \end{nicscolumn*}
  \begin{nicscolumn*}{8.3cm}{1.5cm}{7.5cm}
    \nicsheader{}
    \nicspar{32 bit}
    \nicsitem{Syscall number in \mono{eax}}
    \nicsitem{Arguments: \mono{ebx}, \mono{ecx}, \mono{edx}, \mono{esi}, \mono{edi}, \mono{ebp}}
    \nicsitem{Instruction: \mono{int 0x80}}
    \nicsitem{Result in \mono{eax}}
    \nicsitem{Preserves registers}
  \end{nicscolumn*}
\end{slide}

\begin{slide}{Assembly Workshop\dots}{}
  \begin{nicscolumn*}{1cm}{1.7cm}{14cm}
    \nicsheader{Assembling and linking}
  \end{nicscolumn*}
  \begin{nicscolumn*}{1.5cm}{2.5cm}{5cm}
    \SMALL\nicsverbnoautocenter1
    \begin{nicsterm}
      as -o program.o program.s
      ld -o program program.o
    \end{nicsterm}
  \end{nicscolumn*}
  \begin{nicscolumn*}{7cm}{2.5cm}{5cm}
    \SMALL\nicsverbnoautocenter1
    \begin{nicsterm}
      as --32 -o program.o program.s
      ld -m elf_i386 -o program program.o
    \end{nicsterm}
  \end{nicscolumn*}
  \begin{nicscolumn*}{1.5cm}{3.8cm}{5cm}
    \SMALL\nicsverbnoautocenter1
    \begin{nicsterm}
      yasm -f elf64 program.s
      ld -o program program.o
    \end{nicsterm}
  \end{nicscolumn*}
  \begin{nicscolumn*}{7cm}{3.8cm}{5cm}
    \SMALL\nicsverbnoautocenter1
    \begin{nicsterm}
      yasm -f elf32 program.s
      ld -m elf_i386 -o program program.o
    \end{nicsterm}
  \end{nicscolumn*}
  \begin{nicscolumn*}{1cm}{5.2cm}{14cm}
    \nicsheader{Extracting the code}
  \end{nicscolumn*}
  \begin{nicscolumn*}{1.5cm}{6cm}{10cm}
    \SMALL\nicsverbnoautocenter1
    \begin{nicsterm}
      objcopy -O binary -j .text program program.bin
    \end{nicsterm}
  \end{nicscolumn*}
  \begin{nicscolumn*}{1.5cm}{6.7cm}{10cm}
    \SMALL\nicsverbnoautocenter1
    \begin{nicsterm}
      yasm -f bin program.s -o program.bin
      # BITS 32 or BITS 64 directive is required!
    \end{nicsterm}
  \end{nicscolumn*}
  \begin{nicscolumn*}{10cm}{5.2cm}{14cm}
    \nicsheader{Disassemble}
  \end{nicscolumn*}
  \begin{nicscolumn*}{10.5cm}{6cm}{7cm}
    \SMALL\nicsverbnoautocenter1
    \begin{nicsterm}
      objdump -d
      disassemble  # in gdb
    \end{nicsterm}
  \end{nicscolumn*}
\end{slide}

% TODO(klao): tool slide:
% objdump -d
% gdb disassemble
% readelf ...

% TODO(klao): -g
% Into Makefiles and note on the slides

\section{Stack Smashing}
\nicstitleslide{images/overflow-waterfall}{Stack Buffer Overflows}{Introduction}

\begin{slide}{Stack Buffer Overflows}{Brief history}
  \begin{nicscolumn}
    \nicsitem{\slant{1972} -- US Air Force study}
    % TODO(klao): learn
    \nicsitem{\nicsleadword{\slant{1988} --} Morris worm \\ CERT formed}
    \nicsitem{\slant{1995} -- ``How to Write Buffer Overflows''}
    \nicsitem{\slant{1996} -- ``Stack Smashing''}
    \nicsitem{\slant{2003} -- Blaster worm}
    \kern1cm
    \nicspar{``Memory Corruption Attacks The (almost) Complete History'' \\
      \tiny\url{https://media.blackhat.com/bh-us-10/whitepapers/Meer/BlackHat-USA-2010-Meer-History-of-Memory-Corruption-Attacks-wp.pdf}}
  \end{nicscolumn}
\end{slide}

\begin{slide}{Stack Buffer Overflows}{Brief history}
  \begin{nicscolumn}
    \begin{nicsextern}[width=8cm]{}
      \nicsincludepic{images/mem-corruption-chart}
    \end{nicsextern}
  \end{nicscolumn}
\end{slide}

% TODO(klao): next two slides fix %rbp pointer

\begin{slide}{Stack Frame}{Stack Frame in GCC}
  \begin{nicscolumn*}{3cm}{2cm}{14cm}
    \nicspar{\nicsincludepic[width=10cm]{images/stack-frame-k}}
  \end{nicscolumn*}
\end{slide}

\begin{slide}{Stack Frame}{Buffer Overflow}
  \begin{nicscolumn*}{3cm}{2cm}{14cm}
    \nicspar{\nicsincludepic[width=10cm]{images/stack-frame-bo.pdf}}
  \end{nicscolumn*}
\end{slide}

\begin{slide}{Stack Frame}{Function Prologue and Epilogue}
  \begin{nicscolumn}
    \nicsexterncode{asm}
    \begin{nicsextern}[height=6cm]{}
      push   %rbp          Save frame pointer
      movq   %rsp, %rbp    Initialize frame pointer
      pushq  %rbx          Save non-volatile registers
      pushq  %r12          clobbered by function
      pushq  %r13
      subq   $48, %rsp     Allocate stack frame

      ...                  Function body

      addq   $48, %rsp     Deallocate stack frame
      popq   %r13          Restore registers
      popq   %r12
      popq   %rbx
      leave                Restore %rbp
                           and deallocate stack
      ret
    \end{nicsextern}
  \end{nicscolumn}
\end{slide}

\begin{slide}{Stack Frame}{Function Prologue and Epilogue (Actually)}
  \begin{nicscolumn}
    \nicsexterncode{asm}
    \begin{nicsextern}[height=6cm]{}
      movq   %rbx,-24(%rsp)   Save registers
      movq   %r12,-16(%rsp)
      movq   %r13,-8(%rsp)
      subq   $72, %rsp        Allocate stack frame

      ...                     Function body

      movq   48(%rsp),%rbx    Restore registers
      movq   56(%rsp),%r12
      movq   64(%rsp),%r13
      addq   $72, %rsp        Deallocate stack frame
      ret
    \end{nicsextern}
  \end{nicscolumn}
\end{slide}

\begin{slide}{Stack Frame}{Example}
  \begin{nicscolumn}
    \nicsexterncode{c}
    \begin{nicsextern}[height=5cm]{}
      #include <stdio.h>
      #include <string.h>

      void greet(char *name) {
        char buf[20] = "Hello: ";
        strcat(buf, name);
        puts(buf);
      }
    \end{nicsextern}
  \end{nicscolumn}
\end{slide}

\begin{slide}{Stack Frame}{Example}
  \begin{nicscolumn*}{1cm}{1.2cm}{14cm}
    \nicsexterncode{asm}
    \begin{nicsextern}[height=5.75cm]{}
greet:
        pushl   %ebp
        movl    %esp, %ebp
        subl    $40, %esp
        movl    $0x6c6c6548, -28(%ebp)
        movl    $0x203a6f, -24(%ebp)
        movl    $0, -20(%ebp)
        movl    $0, -16(%ebp)
        movl    $0, -12(%ebp)
        subl    $8, %esp
        pushl   8(%ebp)
        leal    -28(%ebp), %eax
        pushl   %eax
        call    strcat
        addl    $16, %esp
        subl    $12, %esp
        leal    -28(%ebp), %eax
        pushl   %eax
        call    puts
        addl    $16, %esp
        nop
        leave
        ret
    \end{nicsextern}
    \kern3mm
    \begin{nicsterm}
      gcc -m32 -O0 -S -fno-stack-protector -fno-pie
    \end{nicsterm}
  \end{nicscolumn*}
\end{slide}

\begin{slide}{Stack Frame}{Example}
  \begin{nicscolumn*}{1cm}{1.5cm}{14cm}
    \nicsexterncode{asm}
    \begin{nicsextern}[height=5cm]{}
greet:
        pushl   %ebx
        subl    $48, %esp
        movl    $0x6c6c6548, 20(%esp)
        movl    $0x203a6f, 24(%esp)
        movl    $0, 28(%esp)
        movl    $0, 32(%esp)
        movl    $0, 36(%esp)
        pushl   56(%esp)
        leal    31(%esp), %eax
        leal    24(%esp), %ebx
        pushl   %eax
        call    strcpy
        movl    %ebx, (%esp)
        call    puts
        addl    $56, %esp
        popl    %ebx
        ret
    \end{nicsextern}
    \kern5mm
    \begin{nicsterm}
      gcc -m32 -O2 -S -fno-stack-protector -fno-pie
    \end{nicsterm}
  \end{nicscolumn*}
\end{slide}

\begin{slide}{Stack Frame}{Example}
  \begin{nicscolumn*}{1cm}{1.5cm}{14cm}
    \nicsexterncode{asm}
    \begin{nicsextern}[height=5cm]{}
greet:
        pushl   %ebp
        xorl    %eax, %eax
        movl    $3, %ecx
        movl    %esp, %ebp
        pushl   %edi
        leal    -20(%ebp), %edi
        subl    $44, %esp
        movl    $0x6c6c6548, -28(%ebp)
        rep stosl
        leal    -28(%ebp), %edi
        pushl   8(%ebp)
        pushl   %edi
        movl    $0x203a6f, -24(%ebp)
        call    strcat
        movl    %edi, (%esp)
        call    puts
        addl    $16, %esp
        movl    -4(%ebp), %edi
        leave
        ret
    \end{nicsextern}
    \kern5mm
    \begin{nicsterm}
      gcc -m32 -Os -S -fno-stack-protector -fno-pie
    \end{nicsterm}
  \end{nicscolumn*}
\end{slide}

\nicstitleslide{images/stack-smashing}{Stack Smashing}{Workshop}

\begin{slide}{Stack Smashing Workshop}{}
  \begin{nicscolumn}
    \nicsheader{Goal: get first-hand experience}
    \nicsitem{All protections turned off}
    \nicsitem{Explore GCC stack organization on x86-64}
    \nicsitem{Very simple arc injection}
    \nicsitem{Basic shell code}
  \end{nicscolumn}
\end{slide}

\begin{slide}{Buffer Overflows}{Mitigation}
  \begin{nicscolumn}
    \nicsheader{What can you do as a programmer?}
    \nicsitem{Use \mono{std::string}}
    \nicsitem{Use modern, high abstraction C++ (e.g. \mono{std::array})}
    \nicsitem{Use safe variants \mono{strncpy}, \mono{strlcat}, \dots}
    \nicsitem{Aggressively document and check bounds and invariants}
    \nicsitem{Undefined behavior --- \textbf{No!}}
  \end{nicscolumn}
\end{slide}

\begin{slide}{Buffer Overflows}{Mitigation}
  \begin{nicscolumn}
    \nicsheader{Undefined Behavior}
    \nicspar{\just Examples of undefined behavior are memory accesses
      outside of array bounds, signed integer overflow, null pointer
      dereference, modification of the same scalar more than once in
      an expression without sequence points, access to an object
      through a pointer of a different type, etc. Compilers are not
      required to diagnose undefined behavior (although many simple
      situations are diagnosed), and \bold{the compiled program is not
      required to do anything meaningful}.}
    \nicssmallskip
    \nicspar{\SMALL\url{https://en.cppreference.com/w/cpp/language/ub}}
  \end{nicscolumn}
\end{slide}

\begin{slide}{Buffer Overflows}{Mitigation}
  \begin{nicscolumn}
    \nicsheader{SEI CERT standard}
    \nicspar{\SMALL\url{https://wiki.sei.cmu.edu/confluence/display/seccode/SEI+CERT+Coding+Standards}}
    \nicsmedskip
    \nicsitem{Familiarize yourself to know the extent}
    \nicsitem{When in doubt read the rule(s) and weigh the options}
    \nicsbigskip
    \nicsheader{C++ Core Guidlines}
    \nicspar{\SMALL\url{http://isocpp.github.io/CppCoreGuidelines/}}
    \nicsbigskip
    \nicspar{And more\dots}
  \end{nicscolumn}
\end{slide}


\section{Security Mindset}
\nicstitleslide{images/eng_hack}{Security Mindset}{Engineering vs Hacking}

\begin{slide}{Security Mindset}{}
  \begin{nicscolumn}
    \nicsheader{Philosophising}
    \nicsitem{Our brains are faulty}
    \nicsitem{Our models are faulty}
    \nicsitem{Our training is bad}
  \end{nicscolumn}
\end{slide}

\begin{slide}{Security Mindset}{Brains}
  \begin{nicscolumn*}{4cm}{1.5cm}{8cm}
    \begin{nicsextern}{}
      \nicsincludepic{images/overconfidence}
    \end{nicsextern}
  \end{nicscolumn*}
\end{slide}

\begin{slide}{Security Mindset}{Models}
  \begin{nicscolumn}
    \nicsheader{``Secure Perimeter''}
    \nicsitem{is almost always wrong}
  \end{nicscolumn}
\end{slide}

\begin{slide}{Security Mindset}{Training}
  \begin{nicscolumn}
    \nicsheader{Code reviews / Testing}
    \nicsitem{Don't do ``security'' unless you specifically train for that}
  \end{nicscolumn}
\end{slide}

\begin{slide}{Security Mindset}{Brains}
  \begin{nicscolumn}
    \nicsheader{You don't see your own blindspots}
  \end{nicscolumn}
\end{slide}

\begin{slide}{Security Mindset}{Models}
  \begin{nicscolumn}
    \nicsheader{``You are not a target''}
    \nicsitem{is almost always wrong}
  \end{nicscolumn}
\end{slide}

\begin{slide}{Security Mindset}{Training}
  \begin{nicscolumn}
    \nicsheader{Best practices}
    \nicsitem{Don't work without the security mindset}

    \kern3cm
    \nicspar{Bruce Schneier: \href{https://www.schneier.com/blog/archives/2008/03/the_security_mi_1.html}{The Security Mindset}}
  \end{nicscolumn}
\end{slide}

\begin{slide}{Security Mindset}{Example}
  \begin{nicscolumn}
    \nicsexterncode{c}
    \begin{nicsextern}[height=6cm]{}
      #define LOG_INPUT_SIZE 80

      int outputFilenameToLog(char *filename) {
        int success;

        // buffer with size appropriate for logging
        char buf[LOG_INPUT_SIZE];

        // copy filename to buffer
        strcpy(buf, filename);

        // save to log
        success = saveToLog(buf);

        return success;
      }
    \end{nicsextern}
  \end{nicscolumn}
\end{slide}

\begin{slide}{Security Mindset}{Example}
  \begin{nicscolumn}
    \nicsexterncode{c}
    \begin{nicsextern}[height=6cm]{}
      #define LOG_INPUT_SIZE 80

      int outputFilenameToLog(char *filename, int length) {
        int success;

        // buffer with size appropriate for logging
        char buf[LOG_INPUT_SIZE];

        // copy filename to buffer
        strncpy(buf, filename, length);

        // save to log
        success = saveToLog(buf);

        return success;
      }
    \end{nicsextern}
  \end{nicscolumn}
\end{slide}

\begin{slide}{Security Mindset}{Example}
  \begin{nicscolumn}
    \nicsexterncode{c}
    \begin{nicsextern}[height=4cm]{}
      int outputFilenameToLog(char *filename, int length) {
        int success;

        std::string buf = filename;

        // save to log
        success = saveToLog(buf.c_str());

        return success;
      }
    \end{nicsextern}
  \end{nicscolumn}
\end{slide}

\begin{slide}{Security Mindset}{Example}
  \begin{nicscolumn}
    \nicsexterncode{c}
    \begin{nicsextern}[height=6cm]{}
      #define LOG_INPUT_SIZE 80

      int outputFilenameToLog(char *filename, int length) {
        int success;
        char buf[LOG_INPUT_SIZE];

        if (length > LOG_INPUT_SIZE - 1)
          return -1;

        strncpy(buf, filename, length);
        buf[length] = 0;

        success = saveToLog(buf);
        return success;
      }
    \end{nicsextern}
  \end{nicscolumn}
\end{slide}

\begin{slide}{Security Mindset}{Example}
  \begin{nicscolumn}
    \nicsexterncode{c}
    \begin{nicsextern}[height=6cm]{}
      #define LOG_INPUT_SIZE 80

      int outputFilenameToLog(char *filename, int length) {
        int success;
        char buf[LOG_INPUT_SIZE];

        if (length > LOG_INPUT_SIZE - 1)
          return -1;

        strncpy(buf, filename, LOG_INPUT_SIZE - 1);
        buf[LOG_INPUT_SIZE - 1] = 0;

        success = saveToLog(buf);
        return success;
      }
    \end{nicsextern}
  \end{nicscolumn}
\end{slide}

\begin{slide}{Security Mindset}{Brains}
  \begin{nicscolumn}
    \nicsheader{Engineer vs Hacker}
    \nicsitem{Coffee machine story}
  \end{nicscolumn}
\end{slide}


\section{W^X}
\nicstitleslide{images/hardhats}{\mono{W^X}}{Executable-space protection}

\begin{slide}{W^X}{History}
  \begin{nicscolumn}
    \nicsitem{First in OpenBSD in 2003}
    \nicsitem{Windows XP, 2004}
    \nicsitem{Linux}
    \begin{nicsindent}
      \nicsitem{PaX: 2000}
      \nicsitem{Exec Shield: 2003}
    \end{nicsindent}
    \kern4mm
    \nicsitem{NX bit in AMD's 64 bit CPUs}
    \kern4mm
    \nicsitem{Enabled by default in all modern 64bit distros}
    \nicsitem{In 32bit mode enabled with PAE}
  \end{nicscolumn}
\end{slide}

\begin{slide}{W^X}{Explore}
  \begin{nicscolumn}
    \nicsitem{Check binaries and .so's with \mono{readelf -l}}
    \nicsitem{Check running processes: \mono{/proc/<pid>/maps}}
    \kern3mm
    \nicsitem{Our ``target'' from Workshop 2, what if we remove the \mono{-z execstack}}
    \begin{nicsindent}
      \nicsitem{Would arc injection work?}
      \nicsitem{Would shell code work?}
      \nicsitem{How does it fail?}
    \end{nicsindent}
  \end{nicscolumn}
\end{slide}

\begin{slide}{W^X}{Safe?}
  \begin{nicscolumn}
    \nicsheader{How much protection does it provide?}
    \nicssmallskip
    \nicspar{\just Say we find a stack buffer overflow.  We can't run a
      shell code unless we un-\mono{mprotect} it, but we can't just
      call \mono{mprotect} because we can't run arbitrary code.  So,
      we can't do too much right?}
  \end{nicscolumn}
\end{slide}

\begin{slide}{Return Oriented Programming}{}
  \begin{nicscolumn*}{1cm}{2cm}{6cm}
    \tiny\nicsverbnoautocenter1
    \begin{nicscode}
      |        |
      |--------|
      | 0x0765 | §\sans{← Original ret address}
      |--------|
      |        |
    \end{nicscode}
  \end{nicscolumn*}
\end{slide}

\begin{slide}{Return Oriented Programming}{}
  \begin{nicscolumn*}{1cm}{2cm}{6cm}
    \tiny\nicsverbnoautocenter1
    \begin{nicscode}
      |  ...   |
      |--------|
      | 0xgag1 |  §\sans{← Original ret address}
      |    0x0 |
      |--------|
      | 0xg002 |
      |   0x3c |
      |--------|
      | 0xgeg3 |
      |  ...   |
    \end{nicscode}
  \end{nicscolumn*}
  \begin{nicscolumn*}{8cm}{2cm}{6cm}
    \tiny\nicsverbnoautocenter1
    \begin{nicscode}
      0xgag1: pop %rdi
              ret
    \end{nicscode}
    \nicssmallskip
    \begin{nicscode}
      0xg002: pop %rax
              ret
    \end{nicscode}
    \nicssmallskip
    \begin{nicscode}
      0xgeg3: syscall
    \end{nicscode}
  \end{nicscolumn*}
\end{slide}

\begin{slide}{Return Oriented Programming}{Discussion}
  \begin{nicscolumn}
    \nicsitem{All executable memory regions can be used}
    \begin{nicsindent}
      \nicsitem{Main program's \mono{.text}}
      \nicsitem{Shared libraries}
    \end{nicsindent}
    \nicsitem{Not security specific libraries rarely checked for presence of rop gadgets}
  \end{nicscolumn}
\end{slide}

\begin{slide}{Return Oriented Programming}{Demo}
  \begin{nicscolumn}
    \nicspar{Try the \mono{ropper} tool}
  \end{nicscolumn}
\end{slide}

\section{Stack Protector}
\nicstitleslide{images/stack-coins}{Stack Protector}{Detect stack smashing}

\begin{slide}{Stack Protector}{History}
  \begin{nicscolumn}
    \nicsitem{StackGuard: 1997}
    \nicsitem{GCC}
    \begin{nicsindent}
      \nicsitem{2001 -- 2005 ProPolice (IBM)}
      \nicsitem{2005: reimplemented \mono{-fstack-protector} and \mono{-fstack-protector-all}}
      \nicsitem{2012: \mono{-fstack-protector-strong}}
      \nicsitem{On by default in some distros (Arch, Fedora -- strong), but not in others (Debian)}
    \end{nicsindent}
  \end{nicscolumn}
\end{slide}

\begin{slide}{Stack Protector}{Explore}
  \begin{nicscolumn}
    \nicsitem{Check Workshop 2's ``target'' without \mono{-fno-stack-protector}}
    \nicsitem{Check with \mono{-fstack-protector} and \mono{-fstack-protector-all}}
    \nicsitem{Godbolt demo}
  \end{nicscolumn}
\end{slide}

\begin{slide}{Stack Protector}{Discussion}
  \begin{nicscolumn}
    \nicsitem{Only protects the return address}
    \nicsitem{Local function pointers}
    \begin{nicsindent}
      \nicsitem{Tools to rearrange local variables}
    \end{nicsindent}
    \nicsitem{Initialized once for the lifetime of a thread}
  \end{nicscolumn}
\end{slide}


\section{ASLR}
\nicstitleslide{images/dice}{ASLR}{Address Randomization}

\begin{slide}{Address Space Layout Randomization}{History}
  \begin{nicscolumn}
    \nicsitem{Linux}
    \begin{nicsindent}
      \nicsitem{2001: PaX project}
      \nicsitem{2002: full implementation}
      \nicsitem{2005: enabled by default (2.6.12)}
      \nicsitem{????: PIE by default}
    \end{nicsindent}
    \nicsitem{OpenBSD: 2003 first ASLR by default, 2008 first to PIE by default}
    \nicsitem{Windows: 2007, Vista (not for old binaries)}
    \nicsitem{Mac OS X: 2007 system libs, 2011 for all apps}
    \nicsitem{iOS and Android: 2011}
  \end{nicscolumn}
\end{slide}

\begin{slide}{ASLR}{What and how}
  \begin{nicscolumn}
    \nicsitem{What is randomized?}
    \nicsitem{What isn't randomized?}
    \nicsitem{Randomized by how much?}
  \end{nicscolumn}
\end{slide}

\begin{slide}{ASLR}{Explore}
  \begin{nicscolumn}
    \nicsheader{Goal: learn details of ASLR in Linux}
    \nicsitem{Build (or choose) a PIE executable}
    \nicsitem{Observe its memory map over multiple runs}
    \nicsitem{How many independently randomized segments are there?}
    \nicsitem{What are they?}
    \nicsitem{Write your own C or C++ program and check your findings ``from within''}
    \nicsitem{Repeat for 64 and 32 bits}
    \nicssmallskip
    \nicsheader{Additionally}
    \nicsitem{What about non-PIE executables?}
  \end{nicscolumn}
\end{slide}

\begin{slide}{ASLR}{How to defeat}
  \begin{nicscolumn}
    \nicsitem{Reduce entropy}
    \begin{nicsindent}
      \nicsitem{Heap spraying and NOP slides}
    \end{nicsindent}
    \nicsitem{Information leak}
    \begin{nicsindent}
      \nicsitem{Through other vulnerability}
      \nicsitem{Side channel}
      \begin{nicsindent}
        \nicsitem{\href{http://www.cs.ucr.edu/~nael/pubs/micro16.pdf}{Jump Over ASLR}}
        \nicsitem{\href{https://www.vusec.net/projects/anc/}{ASLR ⊕ Cache}}
      \end{nicsindent}
    \end{nicsindent}
    \nicsitem{Brute force}
    \begin{nicsindent}
      \nicsitem{Mitigation: throttle crashes}
    \end{nicsindent}
  \end{nicscolumn}
\end{slide}

\begin{slide}{ASLR}{Linux details}
  \begin{nicscolumn}
    \nicsheader{Performance}
    \nicsitem{On x86-32 there is quite a price (\href{http://ewontfix.com/18/}{read})}
    \nicsitem{x86-64: much better, hardware support for PC-relative addressing, but not free}
    \nicsmedskip
    \nicsheader{Statically linked binaries}
    \nicsitem{Limited ASLR}
    \nicsitem{Build a dynamic binary which has everything statically linked}
  \end{nicscolumn}
\end{slide}

\begin{slide}{ASLR}{Enable/disable}
  \begin{nicscolumn}
    \Small
    \begin{nicsterm}
      echo 0 > /proc/sys/kernel/randomize_va_space
      sysctl -w kernel.randomize_va_space=0
    \end{nicsterm}
    \nicsitem{0: no randomization}
    \nicsitem{1: conservative randomization (shared libs, stack, mmap, VDSO)}
    \nicsitem{2: full randomization (brk() aka. heap too)}
    \nicsmedskip
    \nicspar{\mono{personality} syscall to change execution domain for
      a single process (and its children)}
    \begin{nicsterm}
      setarch `uname -m` -R /bin/bash
    \end{nicsterm}
    \nicsmedskip
    \nicspar{GDB turns it off by default for easier debugging! Turn
      back on with \\ \mono{set disable-randomization off} (or simply
      \mono{aslr on} with GEF and co.)}
  \end{nicscolumn}
\end{slide}

%% End of day 1
%%%%%%%%%%%%%%%%%%%%%%%%%%%%%%%%%%%%%%%%%%%%%%%%%%%%%%%%%%%%%%%%%%%%%%%%%%%%%%%%


\section{Trade-offs}
\nicstitleslide{images/chess}{Security is a Trade-off}{}

\begin{slide}{All or nothing?}{}
  \begin{nicscolumn}
    \nicspar{Security is not all or nothing}
    \nicspar{Have to prioritize it together with}
    \begin{nicsindent}
      \nicsitem{costs}
      \nicsitem{speed of development}
      \nicsitem{maintainability}
      \nicsitem{performance}
      \nicsitem{code quality}
      \nicsitem{\dots}
    \end{nicsindent}
  \end{nicscolumn}
\end{slide}

\begin{slide}{Performance}{}
  \begin{nicscolumn}
    \nicsitem{Many of the general protections are cheap, but not zero
      cost}

    \nicsitem{Example: using \mono{std::string} for all processing
      results in many small strings on heap; might result in
      significant performace hit}
    \nicsitem{Redundant validation}
    \nicsmedskip
    \nicspar{Solution: profile don't guess!}
  \end{nicscolumn}
\end{slide}

\begin{slide}{Development time & costs}{}
  \begin{nicscolumn}
    \nicsitem{Cognitive load can be significant}
    \nicsitem{Solution}
    \begin{nicsindent}
      \nicsitem{Traning. Learn the patterns, become more confident and
        productive, less load}
      \nicsitem{Break things up. Not all phases are equally important}
      \begin{nicsindent}
        \nicsitem{More focus during design}
        \nicsitem{Less during development}
        \nicsitem{More during review and testing}
      \end{nicsindent}
    \end{nicsindent}
    \nicssmallskip
    \nicsitem{Pays off with code quality and maintainability}
    \begin{nicsindent}
      \nicsitem{Biggest enemy: complexity}
    \end{nicsindent}
  \end{nicscolumn}
\end{slide}

\begin{slide}{Flexibility & Usability}{}
  \begin{nicscolumn}
    \nicsitem{Requires a lot of self-discipline}
    \nicsitem{The UX design have to be involved and security minded}
    \nicsitem{Risk modelling and mitigation}
  \end{nicscolumn}
\end{slide}


\section{Heap vulnerabilities}
\nicstitleslide{images/heap-overflow}{Heap Buffer Overflows}{and other vulnerabilities}

\begin{slide}{Heap vulnerabilities}{Overview}
  \begin{nicscolumn}
    \nicsheader{Buffer Over/underflows}
    \nicsitem{Overwrites other data on the heap}
    \nicsitem{Overwrites memory managment internal bookkeeping data}

    \nicsmedskip
    \nicsheader{Improper handling}
    \nicsitem{Use after free}
    \nicsitem{Double \mono{free} / \mono{delete}}
    \nicsitem{Mixing \mono{malloc/new} and \mono{free/delete}}
    \nicsitem{Mixing \mono{new/new[]} and \mono{delete/delete[]}}
    \nicsitem{\dots}
    \nicssmallskip
    \nicsitem{Corrupts memory managment internal bookkeeping data}
  \end{nicscolumn}
\end{slide}

\begin{slide}{Malloc internals}{}
  \begin{nicscolumn*}{1cm}{1.5cm}{14cm}
    \begin{nicsextern}{}
      \nicsincludepic{images/malloc_internals}
    \end{nicsextern}
  \end{nicscolumn*}
  \begin{nicscolumn*}{1cm}{7.2cm}{14cm}
    \nicspar{\url{https://sourceware.org/glibc/wiki/MallocInternals}}
  \end{nicscolumn*}
\end{slide}

\begin{slide}{Malloc internals}{unlink}
  \begin{nicscolumn}
    \begin{nicscode}
      #define unlink(P, BCK, FWD) { \
        FWD = P->fwd; \
        BCK = P->bck; \
        FWD->bck = BCK; \
        BCK->fwd = FWD; \
      }
    \end{nicscode}
  \end{nicscolumn}
\end{slide}

\begin{slide}{Heap vulnerabilities}{Mitigation}
  \begin{nicscolumn}
    \nicsitem{General buffer overflow prevention}
    \nicsitem{Smart pointers}
    \nicsmedskip
    \nicspar{Tools:}
    \nicsitem{Valgrind}
    \nicsitem{GCC & Clang both have superb built-in sanitizers}
    \begin{nicsindent}
      \nicsitem{\mono{-fsanitize=address -fno-omit-frame-pointer}}
      \nicsitem{\url{https://github.com/google/sanitizers/wiki/AddressSanitizerFlags}}
    \end{nicsindent}
  \end{nicscolumn}
\end{slide}

\begin{slide}{Heap vulnerabilities}{Explore}
  \begin{nicscolumn}
    \nicsitem{Does \mono{operator new} calls malloc? How?}
    \nicsitem{What does this code compile to? What are the different parts?}
    {
      \SMALL
      \begin{nicscode}
        int *allocate(int num) {
          return new int[num]();
        }
      \end{nicscode}
    }
    \nicsmedskip

    \nicsitem{Write a program that allocates a few chunks, fills them
      with something easily recognizable, then debug it and
      investigate how it all looks\dots}

  \end{nicscolumn}
\end{slide}

\section{Format String Vulnerabilities}
\nicstitleslide{images/format-string}{Format String Vulnerabilities}{printf(☠)}

\begin{slide}{Format String Vulnerabilities}{}
  \begin{nicscolumn}
    \nicsheader{History}
    \nicsitem{Exploded in 2000}
    \nicsitem{Considered ``bug'', easy to detect}
    \nicsitem{Full remote root in many cases}
  \end{nicscolumn}
\end{slide}

\begin{slide}{Format String Vulnerabilities}{}
  \begin{nicscolumn}
    \nicsheader{What's the problem with this code:}
    \nicsexterncode{c}
    \begin{nicsextern}{}
      char tmpbuf[512];

      snprintf (tmpbuf, sizeof (tmpbuf), "login attempt: %s", user);
      tmpbuf[sizeof (tmpbuf) - 1] = '\0';
      syslog (LOG_NOTICE, tmpbuf);
    \end{nicsextern}
  \end{nicscolumn}
\end{slide}

\begin{slide}{Format String Vulnerabilities}{}
  \begin{nicscolumn}
    \nicsheader{What's the problem with this code:}
    \nicsexterncode{c}
    \begin{nicsextern}{}
      char tmpbuf[512];

      snprintf (tmpbuf, sizeof (tmpbuf), "login attempt: %s", user);
      tmpbuf[sizeof (tmpbuf) - 1] = '\0';
      fprintf (log_file, tmpbuf);
    \end{nicsextern}
  \end{nicscolumn}
\end{slide}

\begin{slide}{Format String Vulnerabilities}{}
  \begin{nicscolumn}
    \nicspar{Basically, this is always wrong and a vulnerability:}
    \nicsmedskip
    \begin{nicscode}
      printf_like_function(attacker_controlled_string);
    \end{nicscode}
  \end{nicscolumn}
\end{slide}

\begin{slide}{Variadic Functions}{}
  \begin{nicscolumn}
    \nicsexterncode{c}
    \begin{nicsextern}[height=6cm]{}
      #include <stdio.h>
      #include <stdarg.h>

      int FindMax(int n, ...) {
        int i, val, largest;
        va_list vl;
        va_start(vl, n);
        largest = va_arg(vl, int);
        for (i = 1; i < n; i++) {
          val = va_arg(vl, int);
          largest = (largest > val) ? largest : val;
        }
        va_end(vl);
        return largest;
      }

      int main() {
        int m;
        m = FindMax(7,702,422,631,834,892,104,772);
        printf("The largest value is: %d\n",m);
        return 0;
      }
    \end{nicsextern}
  \end{nicscolumn}
\end{slide}

\begin{slide}{Format String Vulnerabilities}{How to exploit}
  \begin{nicscolumn}
    \nicsitem{Look arbitrarily up the stack: \mono{"\%p\%p\%p\%p\%p\%p\%p\%p"}}
    \nicsitem{Crash (denial of service): \mono{"\%s\%s\%s\%s\%s\%s\%s\%s"}}
    \nicsitem{Write arbitrary\textsuperscript{*} memory: \mono{\%n}}
    \begin{nicsindent}
      \nicspar{\textsuperscript{*}: with many limitations}
      \nicsitem{Format string \slant{or} buffer we are writing to have to be on the stack (and not too far)}
      \nicsitem{Address can't contain 0 bytes. Makes it significantly harder (if not impossible) on x86-64}
    \end{nicsindent}
  \end{nicscolumn}
\end{slide}

\begin{slide}{Format String Vulnerabilities}{Explore}
  \begin{nicscolumn}
    \nicsitem{Write a small program with which you can explore bad format strings}
    \nicsitem{Walk up the stack, can you identify all the pieces?}
    \nicsitem{Make sure you have stack canaries\dots}
    \begin{nicsindent}
      \nicsitem{What can you tell about stack canary values? Why might that be?}
    \end{nicsindent}
  \end{nicscolumn}
\end{slide}

\begin{slide}{Format String Vulnerabilities}{Mitigation}
  \begin{nicscolumn}
    \nicsitem{Format strings should be constant literals}
    \begin{nicsindent}
      \nicsitem{OK, \slant{maybe} allow \mono{_("format")} with \mono{gettext}}
    \end{nicsindent}
    \nicsitem{Static analysis tools detect these well}
    \nicsmedskip
    \nicspar{Generally for variadic functions}
    \nicsitem{Don't use variadic functions}
    \nicsitem{Use variadic templates}
  \end{nicscolumn}
\end{slide}

\section{Integers}
\nicstitleslide{images/integers}{Integer Vulnerabilities}{``Signed overflow is Undefined Behavior?''}

\begin{slide}{Integer Vulnerabilities}{}
  \begin{nicscolumn}
    \nicsitem{Integer conversion/promotion rules are \slant{complex}}
    \nicssmallskip
    \begin{nicsindent}
    \nicspar{\just ``Well, at the time we thought compatibility with B
      was important, so we kept that; I wish we hadn’t, no one cares
      about B anymore.''}
    \end{nicsindent}

    \nicsmedskip
    \nicsitem{Many things are ``Implementation Defined''}
    \begin{nicsindent}
      \nicsitem{Can cause problems much later than the code was written}
    \end{nicsindent}

    \nicsmedskip
    \nicsitem{Hard to reason about}
    \begin{nicsindent}
      \nicsitem{\mono{-INT_MIN} is \mono{INT_MIN}?}
    \end{nicsindent}
  \end{nicscolumn}
\end{slide}

\begin{slide}{Integer Vulnerabilities}{Examples}
  \begin{nicscolumn}
    \nicsexterncode{c}
    \begin{nicsextern}[height=4cm]{}
      void getComment(size_t len, char *src) {
        size_t size;
        size = len - 2;
        char *comment = (char *)malloc(size + 1);
        memcpy(comment, src, size);
        return;
      }
    \end{nicsextern}
  \end{nicscolumn}
\end{slide}

\begin{slide}{Integer Vulnerabilities}{Examples}
  \begin{nicscolumn}
    \begin{nicscode}
      p = calloc(sizeof(element_t), count);
    \end{nicscode}
  \end{nicscolumn}
\end{slide}

\begin{slide}{Integer Vulnerabilities}{Examples}
  \begin{nicscolumn}
    \begin{nicscode}
      if (off > len - sizeof(type_name)) goto error;
    \end{nicscode}
  \end{nicscolumn}
\end{slide}

\begin{slide}{Integer Vulnerabilities}{Examples}
  \begin{nicscolumn}
    \nicsexterncode{c}
    \begin{nicsextern}[height=4cm]{}
      void initialize_array(int size) {
        if (size < MAX_ARRAY_SIZE) {
          array = malloc(size);
          /* initialize array */
        } else {
          /* handle error */
        }
      }
    \end{nicsextern}
  \end{nicscolumn}
\end{slide}

\begin{slide}{Integer Vulnerabilities}{Mitigation}
  \begin{nicscolumn}
    \nicsitem{Check the ranges obsessively}
    \nicsitem{Use \mono{size_t} for sizes of objects}
    \begin{nicsindent}
      \nicsitem{Or at least \mono{unsigned} types}
      \nicsitem{C11 Annex K: \mono{rsize_t}}
    \end{nicsindent}
    \nicsitem{User defined types with better semantics}
    \begin{nicsindent}
      \nicsitem{User defined literals!}
      \nicsitem{\mono{5_hour + 10_kg}}
    \end{nicsindent}
  \end{nicscolumn}
\end{slide}

\section{FORTIFY_SOURCE}
\nicstitleslide{images/fortification}{FORTIFY_SOURCE}{Fixing your buffer overflows for you}

\section{RELRO}
\nicstitleslide{images/ro-maci}{RELRO}{RELocation Read-Only}

\section{CET}
\nicstitleslide{images/control-flow}{Control-Flow Enforcement Technology}{}

\begin{slide}{Control-Flow Enforcement Technology}{}
  \begin{nicscolumn}
    \nicsitem{\href{https://software.intel.com/sites/default/files/managed/4d/2a/control-flow-enforcement-technology-preview.pdf}{Intel Spec}}
    \nicsitem{\url{https://lwn.net/Articles/758245/}}
  \end{nicscolumn}
\end{slide}

\section{Server exploit}
\nicstitleslide{images/workshop}{Exploit a ``Server''}{Workshop}

\begin{slide}{``Server'' exploit}{}
  \begin{nicscolumn}
    \nicspar{Goal: to see a (reasonably) real-world example with all
      of the protections enabled, which can nevertheless be
      exploited.}
  \end{nicscolumn}
\end{slide}



\section{Best Practices}
\nicstitleslide{images/best-practices}{Best Practices}{What to do}

\begin{slide}{Best Practices}{}
  \begin{nicscolumn}
    \nicspar{\just Within C++, there is a much Small and cleaner
      language struggling to get out.}
    \nicspar{~\kern8cm Bjarne Stroustrup}
  \end{nicscolumn}
\end{slide}

\begin{slide}{Best Practices}{}
  \begin{nicscolumn}
    \nicspar{\just Within C++, there is a much Small and cleaner
      language struggling to get out.}
    \nicspar{~\kern8cm Bjarne Stroustrup}
    \nicsmedskip
    \nicspar{\just And no, that Small and cleaner language is not Java
      or C\#.}
  \end{nicscolumn}
\end{slide}

\begin{slide}{Best Practices}{}
  \begin{nicscolumn}
    \nicsheader{SEI CERT standard}
    \nicspar{\SMALL\url{https://wiki.sei.cmu.edu/confluence/display/seccode/SEI+CERT+Coding+Standards}}
    \nicssmallskip
    \nicsitem{Familiarize yourself to know the extent}
    \nicsitem{When in doubt read the rule(s) and weigh the options}
    \nicsmedskip
    \nicsheader{C++ Core Guidlines}
    \nicspar{\SMALL\url{http://isocpp.github.io/CppCoreGuidelines/}}
    \nicsmedskip
    \nicspar{And more\dots}
  \end{nicscolumn}
\end{slide}

\begin{slide}{Best Practices}{}
  \begin{nicscolumn}
    \nicspar{Run tests with many different setups}
    \nicsitem{Different compilers}
    \nicsitem{Different optimization options}
    \nicsitem{Memory constrained (?)}
  \end{nicscolumn}
\end{slide}

\begin{slide}{Best Practices}{}
  \begin{nicscolumn}
    \nicspar{Not allocate with new}
    \nicsitem{Smart pointers}
  \end{nicscolumn}
\end{slide}

\begin{slide}{Best Practices}{}
  \begin{nicscolumn}
    \nicspar{Not allocate with new}
    \nicsitem{Smart pointers}
    \nicsmedskip
    \nicspar{Shared pointers stink}
    \nicsitem{If they start to spread --- time to think}
  \end{nicscolumn}
\end{slide}

\begin{slide}{Best Practices}{}
  \begin{nicscolumn}
    \nicsitem{Prefer \mono{enum class} to \mono{enum}s}
    \begin{nicsindent}
      \nicsitem{No undefined behavior of converting \mono{int} to enum out of range}
    \end{nicsindent}
  \end{nicscolumn}
\end{slide}

\begin{slide}{Best Practices}{}
  \begin{nicscolumn}
    \nicsitem{String literals: \mono{"foobar"s}}
    \nicsmedskip
    \nicsitem{User defined literals: \mono{5_hour + 10_kg}}
  \end{nicscolumn}
\end{slide}

\begin{slide}{Best Practices}{}
  \begin{nicscolumn}
    \nicsitem{Check the ranges obsessively}
    \nicsitem{Use \mono{size_t} for sizes of objects}
    \begin{nicsindent}
      \nicsitem{Or at least \mono{unsigned} types}
      \nicsitem{C11 Annex K: \mono{rsize_t}}
    \end{nicsindent}
  \end{nicscolumn}
\end{slide}

\begin{slide}{C11, Annex K}{}
  \begin{nicscolumn}
    \nicspar{Secure variants of many functions}
    \nicsitem{\mono{memcpy_s}, \mono{strcat_s}, \dots}
    \nicsitem{\mono{gets_s}, \mono{printf_s}, \dots}
    \nicsmedskip
    \nicsitem{Not in glibc}
    \nicsitem{Generally poor adoption}
    \begin{nicsindent}
      \nicsitem{Considered for removal}
      \nicsitem{Interface & ease of adoption matters}
      \nicsitem{\href{http://www.open-std.org/jtc1/sc22/wg14/www/docs/n1967.htm}{Field Experience With Annex K — Bounds Checking Interfaces}}
    \end{nicsindent}
  \end{nicscolumn}
\end{slide}

\begin{slide}{C11, Annex K}{Continued}
  \begin{nicscolumn}
    \nicspar{\mono{memset} is sometimes used to zero-out sensitive memory}
    \nicsitem{Often optimized out if the memory ``cannot'' be accessed later}
    \nicsitem{\mono{memset_s} is specified to \slant{not be} optimized out}
    \nicsitem{Research alternatives on your platform if you need this}
  \end{nicscolumn}
\end{slide}

\begin{slide}{Compilers are Evil}{Or maybe specs are}
  \begin{nicscolumn}
    \nicsexterncode{c}
    \begin{nicsextern}[height=4cm]{}
      void GetData(char *MFAddr) {
        char pwd[64];
        if (GetPasswordFromUser(pwd, sizeof(pwd))) {
          ...
        }
        memset(pwd, 0, sizeof(pwd));
      }
    \end{nicsextern}
  \end{nicscolumn}
\end{slide}

\begin{slide}{Undefined Behavior is Evil}{Or maybe compilers are}
  \begin{nicscolumn}
    \nicsexterncode{c}
    \begin{nicsextern}[height=6cm]{}
      #include <iostream>
      #include <complex>
      using namespace std;

      int main(void) {
        complex<int> delta;
        complex<int> mc[4] = {0};

        for (int di = 0; di < 4; di++, delta = mc[di]) {
          cout << di << endl;
        }
      }
    \end{nicsextern}
  \end{nicscolumn}
\end{slide}

\begin{slide}{Best Practices}{}
  \begin{nicscolumn}
    \nicsheader{Warnings are errors}
  \end{nicscolumn}
\end{slide}

\begin{slide}{Best Practices}{}
  \begin{nicscolumn}
    \nicspar{Be selective with the libraries you use}
    \nicsitem{Open source libraries have more eyes on them, but also easier
      targets}
  \end{nicscolumn}
\end{slide}

\begin{slide}{Dynamic Analysis}{}
  \begin{nicscolumn}
    \nicspar{GCC and Clang both have good sanitizers}
    \begin{nicsindent}
      \nicsitem{asan}
      \nicsitem{tsan}
      \nicsitem{ubsan}
    \end{nicsindent}
    \nicsmedskip
    \nicspar{Espesically effective with stress and scalability tests}
  \end{nicscolumn}
\end{slide}

\begin{slide}{Best Practices}{}
  \begin{nicscolumn}
    \nicsheader{Do stress testing}
  \end{nicscolumn}
\end{slide}

\begin{slide}{Fuzz Testing}{}
  \begin{nicscolumn}
    \nicspar{Fuzz testing can be very effective in uncovering bugs
      which might lead to vulnerabilities}
    \nicsitem{Requires a significant investment (similar to TDD)}
  \end{nicscolumn}
\end{slide}


\section{Miscellaneous}
\nicstitleslide{images/misc}{Miscellaneous}{Topics we didn't touch}

\begin{slide}{Concurrency}{}
  \begin{nicscolumn}
    \nicsitem{A world in its own}
    \nicsmedskip
    \nicsitem{Minimize shared data}
    \nicsitem{Thread sanitizer}
  \end{nicscolumn}
  \begin{nicscolumn*}{8cm}{2cm}{7cm}
    \begin{nicsextern}{}
      \nicsincludepic{images/race}
    \end{nicsextern}
  \end{nicscolumn*}
\end{slide}

\begin{slide}{Unicode}{Complexity in strings}
  \begin{nicscolumn}
    \nicsitem{Unicode Standard is more than 1000 pages}
    \nicsitem{Many things that seem easy are actually quite complex}
    \begin{nicsindent}
      \nicsitem{Are two strings ``equal''?}
      \nicsitem{Different strings can still look exactly the same}
      \nicsitem{Length of a string?}
    \end{nicsindent}

    \nicsbigskip
    \nicspar{I18N in general is tricky and common source of bugs & vulnerabilities}
  \end{nicscolumn}
\end{slide}

\begin{slide}{Sensitive Information}{}
  \begin{nicscolumn}
    \nicsitem{Logging is not neutral}
    \begin{nicsindent}
      \nicsitem{Memory dumps\dots}
    \end{nicsindent}
    \nicsitem{Lifetime management}
    \nicsitem{What gets on a disk}
  \end{nicscolumn}
\end{slide}

\begin{slide}{Secure communication}{}
  \begin{nicscolumn}
    \nicsitem{DDoS, Men-in-the-middle, Spoofing, Byzantine failures, Side channels, oh my!}

    \nicsbigskip
    \nicsitem{Cryptography in general\dots}
  \end{nicscolumn}
\end{slide}


\nicstitleslide{images/thankyou}{The End}{Thanks, and good luck with all your coding!}

\end{document}
